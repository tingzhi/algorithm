\documentclass[11pt]{scrreprt}
\usepackage[utf8]{inputenc}
\usepackage{graphicx}
\usepackage{listings} %插入代码
\usepackage{xcolor} %代码高亮
%\usepackage[demo]{graphicx}
\usepackage{caption}
\usepackage{tabularx}
\usepackage[labelformat=simple]{subcaption}
\usepackage{amsmath}

\renewcommand\thesubfigure{(\alph{subfigure})} % see subcaption doc

\lstset{numbers=left, %设置行号位置
	numberstyle=\tiny, %设置行号大小
	keywordstyle=\color{blue}, %设置关键字颜色
	commentstyle=\color[cmyk]{1,0,1,0}, %设置注释颜色
	frame=single, %设置边框格式
	escapeinside=``, %逃逸字符(1左面的键),用于显示中文
	breaklines, %自动折行
	extendedchars=false, %解决代码跨页时,章节标题,页眉等汉字不显示的问题
	xleftmargin=2em,xrightmargin=2em, aboveskip=1em, %设置边距
	tabsize=4, %设置tab空格数
	showspaces=false %不显示空格
}

\title{\textbf{Project 3 Report}}
\subtitle{CS325 Analysis of Algorithms, Summer 2015}
\author{\textsf{\textbf{Project Group 1}}\\
		\textsf{Tingzhi Li}\\
		\textsf{Nicholas Nelson}\\
		\textsf{Chunyang Zhang}}
\date{}

\begin{document}
\maketitle

\chapter{Transshipment Model}
\section{Part A}
The transshipment model is an extension of the transportation model. 
In addition to the standard transportation model, there are now 
intermediate transshipment points added between the sources (plants) 
and destinations (retailers). Items being shipped from a Plant 
($p_i$) must be shipped to a Warehouse ($w_j$) before being shipped 
to the Retailer ($r_k$). Each Plant will have an associated supply 
($s_i$) and each Retailer will have a demand ($d_k$). The number of 
plants is $n$, number of warehouses is $q$ and the number of 
retailers is $m$. The edges $(i,j)$ from plant ($p_i$) to warehouse 
($w_j$) have associated costs denoted $cp(j,k)$.

\subsection{i. Linear Formulation}

The objective function and constraints for this problem are as 
follows:

\begin{itemize}
	\item Minimize
[$10x_{1,1}+15x_{1,2}+11x_{2,1}+8x_{2,2}+13x_{3,1}+8x_{3,2}+9x_{3,3}+14x_{4,2}+8x_{4,3}+5y_{1,1}+6y_{1,2}+7y_{1,3}+10y_{1,4}+12y_{2,3}+8y_{2,4}+10y_{2,5}+14y_{2,6}+14y_{3,4}+12y_{3,5}+12y_{3,6}+6y_{3,7}$] 
where $x_{i,j}$ represents number of refrigerators to be shipped 
from plant $i$ to warehouse $j$, and $y_{j,k}$ represents number of 
refrigerators to be shipped  from warehouse $j$ to retailer $k$.
	\item Supply constraints:
	\begin{enumerate}
		\item $x_{1,1} + x_{1,2} <= 150$
		\item $x_{2,1} + x_{2,2} <= 450$
		\item $x_{3,1} + x_{3,2} + x_{3,3} <= 250$
		\item $x_{4,2} + x_{4,3} <= 150$
	\end{enumerate}
	\item Retailer constraints:
	\begin{enumerate}
		\item $y_{1,1} >= 100$
		\item $y_{1,2} >= 150$
		\item $y_{1,3} + y_{2,3} >= 100$
		\item $y_{1,4} + y_{2,4} + y_{3,4} >= 200$
		\item $y_{2,5} + y_{3,5} >= 200$
		\item $y_{2,6} + y_{3,6} >= 150$
		\item $y_{3,7} >= 100$
	\end{enumerate}
	\item The number of refrigerators shipped into any individual 
		warehouse has to be equal to the number of refrigerators 
		shipped out from that warehouse, thus constrained by:
	\begin{enumerate}
		\item $x_{1,1} + x_{2,1} + x_{3,1} - y_{1,1} - y_{1,2} - y_{1,3} - y_{1,4} = 0$
		\item $x_{1,2} + x_{2,2} + x_{3,2} + x_{4,2} - y_{2,3} - y_{2,4} - y_{2,5} - y_{2,6} = 0$
		\item $x_{3,3} + x_{4,3} - y_{3,4} - y_{3,5} - y_{3,6} - y_{3,7} = 0$
	\end{enumerate}
\end{itemize}

\subsection{ii. Optimal Solution and Linear Program}

The optimal solution for this problem is as follows:\\

Objective value: 17100\\

\begin{tabular}{|c|c|c|c|c|c|c|c|c|c|}
	\hline variables & $x_{1,1}$   &  $x_{1,2}$ & $x_{2,1}$ & $x_{2,2}$ & $x_{3,1}$ & $x_{3,2}$ & $x_{3,3}$ & $x_{4,2}$ & $x_{4,3}$      \\
	\hline optimal value & 150  &  0 & 200 & 250 & 0 & 150 & 100 & 0 & 150            \\
	\hline
\end{tabular} \\

\begin{tabular}{|c|c|c|c|c|c|c|c|c|c|c|c|c|}
	\hline variables & $y_{1,1}$ & $y_{1,2}$ & $y_{1,3}$ & $y_{1,4}$ & $y_{2,3}$ & $y_{2,4}$ & $y_{2,5}$ & $y_{2,6}$ & $y_{3,4}$ & $y_{3,5}$ & $y_{3,6}$ & $y_{3,7}$  \\
	\hline optimal value & 	100 & 150 & 100 & 0 & 0 & 200 & 200 & 0 & 0 & 0 & 150 & 100 \\
	 \hline
\end{tabular} \\

Lindo Code
\begin{lstlisting}[basicstyle=\small,language=c]
min 10x11+15x12+11x21+8x22+13x31+8x32+9x33+14x42+8x43+5y11+6y12+7y13+10y14+12y23+8y24+10y25+14y26+14y34+12y35+12y36+6y37

ST
x11+x12<=150
x21+x22<=450
x31+x32+x33<=250
x42+x43<= 150

y11>=100
y12>=150
y13+y23>=100
y14+y24+y34>=200
y25+y35>=200
y26+y36>=150
y37>=100

x11+x21+x31-y11-y12-y13-y14=0
x12+x22+x32+x42-y23-y24-y25-y26=0
x33+x43-y34-y35-y36-y37=0
\end{lstlisting}

\subsection{iii. Optimal Shipping Routes/Minimum Costs}
The optimal shipping routes, based upon minimizing costs, are as follows: \\

\begin{tabular}{|c|c|}
	\hline From plants to warehouses & From warehouses to retailers \\
	\hline $x_{1,1} = 150$ & $y_{1,1} = 100$ \\
	\hline $x_{2,1} = 200$ &$y_{1,2} = 150$ \\
	\hline $x_{2,2} = 250$ &$y_{1,3} = 100$ \\
	\hline $x_{3,2} = 150$ &$y_{2,4} = 200$ \\
	\hline $x_{3,3} = 100$ & $y_{2,5} = 200$  \\
	\hline $x_{4,3} = 150$ & $y_{3,6} = 150$  \\
	\hline & $y_{3,7} = 100$ \\
	\hline
\end{tabular} \\

Minimum cost = \$17,100

\section{Part B}
Due to old infrastructure, warehouse 2 is going to close eliminating 
all of the associated routes. Therefore, all calculations as to the
optimal shipping route and minimized costs must be re-evaluated. The 
feasability of shipping all refrigerators to either warehouse 1 or 3
and then to the retailers without using warehouse 2 must also be
evaluated.

\subsection{Evaluation}
Without warehouse 2, there is no optimal solution to the transshipment 
model. It is not feasible to ship all refrigerators to either warehouse
1 or 3 and then to the retailers without using warehouse 2. Even if 
plants $p_3$ and $p_4$ shipped all refrigerators to warehouse $w_3$, 
the demands from $r_5$, $r_6$ and $r_7$ would not be met because
$p_3 + p_4 < r_5 + r_6 + r_7$.

\section{Part C}
Instead of closing warehouse 2, management has decided to keep a 
portion of it open but limit shipments to 100 refrigerators per week. 
The feasability of this model must be evaluated, and an optimal 
solution determined.

\subsection{Evaluation}
With warehouse 2 limited to 100 refrigerators per week, the model is
feasible. In contrast to the code we wrote in Part $A$, we only have to add one constrain line that $x_{1,2}$+$x_{2,2}$+$x_{3,2}$+$x_{4,2}$$ <= 100$. Then the optimal shipping routes, based upon minimizing costs, 
are as follows: \\

\begin{tabular}{|c|c|}
	\hline From plants to warehouses & From warehouses to retailers \\
	\hline $x_{1,1} = 150$ & $y_{1,1} = 100$ \\
	\hline $x_{2,1} = 350$ & $y_{1,2} = 150$ \\
	\hline $x_{2,2} = 100$ & $y_{1,3} = 100$  \\
	\hline $x_{3,3} = 250$ & $y_{1,4} = 150$  \\
	\hline $x_{4,3} = 150$ & $y_{2,4} = 50$  \\
	\hline 				   & $y_{2,5} = 50$ \\
	\hline 				   & $y_{3,5} = 150$ \\
	\hline 				   & $y_{3,6} = 150$ \\
	\hline 				   & $y_{3,7} = 100$ \\
	\hline
\end{tabular} \\

Minimum cost = \$18300

\section{Part D}
The generalized linear programming model for the transshipment problem 
can be expressed using the following formulation, objective function,
and constraints:\\

Objective function:
\begin{displaymath}
\min \left(\sum_{i=1}^{n} \sum_{j=1}^{q} cp(i,j)x_{ij} + \sum_{j=1}^{q} \sum_{k=1}^{m} cw(j,k)y_{jk}\right)
\end{displaymath}

$x_{ij}=0 \textrm{ if } cp(i,j) \textrm{ is not applicable, }$

$y_{jk}=0 \textrm{ if } cw(j,k) \textrm{ is not applicable,}$

where $x_{ij}$ represents the number of refrigerators to be shipped 
from plant $i$ to warehouse $j$ and $y_{jk}$ represents number of 
refrigerators to be shipped from warehouse $j$ to retailer $k$.\\

The constraints for this problem are as follows:

\begin{itemize}
	\item Supply constraints:
	\begin{displaymath}
	\sum_{j=1}^{q} x_{ij} \leq s_i 
	\qquad\textrm{where } i=1,2,\cdots,n
	\end{displaymath}
	\item Retailer constraints:
	\begin{displaymath}
	\sum_{j=1}^{q} y_{jk} \geq d_k 
	\qquad\textrm{where } k=1,2,\cdots,m
	\end{displaymath}
	\item The number of refrigerators shipped into any individual 
	warehouse has to be equal to the number of refrigerators 
	shipped out from that warehouse, thus constrained by:
	\begin{displaymath}
	\sum_{i=1}^{n} x_{ij} = \sum_{k=1}^{m} y_{jk} 
	\qquad\textrm{where } j=1,2,\cdots,q
	\end{displaymath}
\end{itemize}




\chapter{A Mixture Problem}

\section{Part A}
Veronica the owner of Very Veggie Vegeria is creating a new 
healthy salad that is low in calories but meets certain 
nutritional requirements. A salad is any combination of the 
following ingredients:\\

Tomato, Lettuce, Spinach, Carrot, Smoked Tofu, Sunflower Seeds, Chickpeas,
Oil\\

Each salad must contain:

\begin{itemize}
	\item At least 15 grams of protein
	\item At least 2 and at most 8 grams of fat
	\item At least 4 grams of carbohydrates
	\item At most 200 milligrams of sodium
	\item At least 40\% leafy greens by mass
\end{itemize}

The nutritional contents of these ingredients (per 100 grams)
and cost are:

\begin{figure}[!htbp]
	\includegraphics[width=1.0\textwidth]{nutritional.png}
\end{figure}

\subsection{i. Linear Formulation}
The objective function and constraints for this problem are as 
follows:

\begin{itemize}
	\item Minimize
[$21x_{1}+16x_{2}+40x_{3}+41x_{4}+120x_{5}+585x_{6}+164x_{7}+884x_{8}$]
where $x_{i}$ represents the weight (in kgram units) for ingredient $i$.
	\item Constraints
	\begin{enumerate}
		\item At least 15 grams of protein:\\
		$0.85x_{1}+1.62x_{2}+2.86x_{3}+0.93x_{4}+23.4x_{5}+16.00x_{6}+9.00x_{7} >= 15$
		\item At least 2 grams of fat:\\
		$0.33x_{1}+0.20x_{2}+0.39x_{3}+0.24x_{4}+48.7x_{5}+5.00x_{6}+2.6x_{7}+100.00x_{8} >= 2$
		\item At most 8 grams of fat:\\
		$0.33x_{1}+0.20x_{2}+0.39x_{3}+0.24x_{4}+48.7x_{5}+5.00x_{6}+2.6x_{7}+100.00x_{8} <= 8$
		\item At least 4 grams of carbohydrates:\\
		$4.64x_{1}+2.37x_{2}+3.63x_{3}+9.58x_{4}+15.00x_{5}+3.00x_{6}+27.0x_{7} >= 4$
		\item At most 200 milligrams of sodium:\\
		$9.00x_{1}+28.00x_{2}+65.00x_{3}+69.00x_{4}+3.80x_{5}+120.00x_{6}+78.00x_{7} <= 200$
		\item At least 40\% leafy greens by mass:\\
		$x_{2}+x_{3}-0.4x_{1}-0.4x_{2}-0.4x_{3}-0.4x_{4}-0.4x_{5}-0.4x_{6}-0.4x_{7}-0.4x_{8} >= 0$
	\end{enumerate}
\end{itemize}

\subsection{ii. Optimal Solution and Linear Program}
The optimal solution for this problem is as follows:\\

Objective value: 114.7541\\

\begin{tabular}{|c|c|c|c|c|c|c|c|c|}
	\hline variables & $x_{1}$ & $x_{2}$ & $x_{3}$ & $x_{4}$ & $x_{5}$ & $x_{6}$ & $x_{7}$ & $x_{8}$ \\
	\hline optimal value & 0 & 58.54801 & 0 & 0 & 0 & 87.82201 & 0 & 0 \\
	\hline
\end{tabular} \\

{\it Note:} The units for the objective value are kcal and the variables are in grams.\\

Lindo Code
\begin{lstlisting}[basicstyle=\small,language=c]
min 21x1+16x2+40x3+41x4+585x5+120x6+164x7+884x8
ST
0.85x1+1.62x2+2.86x3+0.93x4+23.4x5+16.00x6+9.00x7 >= 15
0.33x1+0.20x2+0.39x3+0.24x4+48.7x5+5.00x6+2.6x7+100.00x8 >= 2
0.33x1+0.20x2+0.39x3+0.24x4+48.7x5+5.00x6+2.6x7+100.00x8 <= 8
4.64x1+2.37x2+3.63x3+9.58x4+15.00x5+3.00x6+27.0x7 >= 4
9.00x1+28.00x2+65.00x3+69.00x4+3.80x5+120.00x6+78.00x7 <= 200
x2+x3-0.4x1-0.4x2-0.4x3-0.4x4-0.4x5-0.4x6-0.4x7-0.4x8 >= 0
\end{lstlisting}

\subsection{iii. Cost of Low Calorie Salad}
The cost of the low calorie salad, based upon minimizing calories, is as follows:

\begin{displaymath}
({\$0.75 \over 100\text{g}})x_2 + ({\$2.15 \over 100\text{g}})x_6 = (\$0.0075 \times 58.54801\text{g}) + (\$0.0215 \times 87.82201\text{g}) = \$2.32728329
\end{displaymath}

Therefore, the cost of the low calorie salad is approximately \$2.33.

\section{Part B}
Veronica realizes that it is also important to minimize the cost associated with the new salad. Unfortunately some of the ingredients can be expensive. Therefore, we must determine the combination of ingredients that minimizes cost.

\subsection{i. Linear Formulation}
The objective function and constraints for this problem are as 
follows:

\begin{itemize}
	\item Minimize
[$1.00x_{1}+0.75x_{2}+0.50x_{3}+0.50x_{4}+0.45x_{5}+2.15x_{6}+0.95x_{7}+2.00x_{8}$] where $x_{i}$ represents the cost (in dollars) per 100 grams of ingredient $i$.
	\item Constraints
	\begin{enumerate}
		\item At least 15 grams of protein:\\
		$0.85x_{1}+1.62x_{2}+2.86x_{3}+0.93x_{4}+23.4x_{5}+16.00x_{6}+9.00x_{7} >= 15$
		\item At least 2 grams of fat:\\
		$0.33x_{1}+0.20x_{2}+0.39x_{3}+0.24x_{4}+48.7x_{5}+5.00x_{6}+2.6x_{7}+100.00x_{8} >= 2$
		\item At most 8 grams of fat:\\
		$0.33x_{1}+0.20x_{2}+0.39x_{3}+0.24x_{4}+48.7x_{5}+5.00x_{6}+2.6x_{7}+100.00x_{8} <= 8$
		\item At least 4 grams of carbohydrates:\\
		$4.64x_{1}+2.37x_{2}+3.63x_{3}+9.58x_{4}+15.00x_{5}+3.00x_{6}+27.0x_{7} >= 4$
		\item At most 200 milligrams of sodium:\\
		$9.00x_{1}+28.00x_{2}+65.00x_{3}+69.00x_{4}+3.80x_{5}+120.00x_{6}+78.00x_{7} <= 200$
		\item At least 40\% leafy greens by mass:\\
		$x_{2}+x_{3}-0.4x_{1}-0.4x_{2}-0.4x_{3}-0.4x_{4}-0.4x_{5}-0.4x_{6}-0.4x_{7}-0.4x_{8} >= 0$
	\end{enumerate}
\end{itemize}


\subsection{ii. Optimal Solution and Linear Program}
The optimal solution for this problem is as follows:\\

Objective value: 1.554133\\

\begin{tabular}{|c|c|c|c|c|c|c|c|c|}
	\hline variables & $x_{1}$ & $x_{2}$ & $x_{3}$ & $x_{4}$ & $x_{5}$ & $x_{6}$ & $x_{7}$ & $x_{8}$ \\
	\hline optimal value & 0 & 0 & 83.22983 & 0 & 9.608330 & 0 & 115.2364 & 0 \\
	\hline
\end{tabular} \\

{\it Note:} The units for the objective value are dollars and  the variables are in grams.\\

\pagebreak
Lindo Code
\begin{lstlisting}[basicstyle=\small,language=c]
min 1.00x1+0.75x2+0.50x3+0.50x4+0.45x5+2.15x6+0.95x7+2.00x8
ST
0.85x1+1.62x2+2.86x3+0.93x4+23.4x5+16.00x6+9.00x7 >= 15
0.33x1+0.20x2+0.39x3+0.24x4+48.7x5+5.00x6+2.6x7+100.00x8 >= 2
0.33x1+0.20x2+0.39x3+0.24x4+48.7x5+5.00x6+2.6x7+100.00x8 <= 8
4.64x1+2.37x2+3.63x3+9.58x4+15.00x5+3.00x6+27.0x7 >= 4
9.00x1+28.00x2+65.00x3+69.00x4+3.80x5+120.00x6+78.00x7 <= 200
x2+x3-0.4x1-0.4x2-0.4x3-0.4x4-0.4x5-0.4x6-0.4x7-0.4x8 >= 0
\end{lstlisting}

\subsection{iii. Calories of Low Cost Salad}

The calories of the low cost salad, based upon minimizing costs, is as follows:

\begin{displaymath}
({40\text{ kcal} \over 100\text{g}})x_3 + ({585\text{ kcal} \over 100\text{g}})x_5 + ({164\text{ kcal} \over 100\text{g}})x_7 =
\end{displaymath}
\begin{displaymath}
(0.4\text{ kcal} \times 83.22983\text{g}) + (5.85\text{ kcal} \times 9.608330\text{g}) + (1.64\text{ kcal} \times 115.2364\text{g}) =
\end{displaymath}
\begin{displaymath}
278.4883585\text{ kcal}
\end{displaymath}

Therefore, the calories of the low cost salad are approximately 278.50 kcal.

\section{Part C}

\subsection{i. Possible Methods}
We found the following list of possible methods to determine a solution that would meet all conditions and constraints:
\begin{itemize}
	\item Minimize calories, using the additional constraints that calories must be less than or equal to 250 kcal and cost must be less than or equal to \$2.00.
	\item Minimize cost, using the additional constraints that calories must be less than or equal to 250 kcal and cost must be less than or equal to \$2.00.
	\item Minimize the weight of ingredients, using the additional constraints that calories must be less than or equal to 250 kcal and cost must be less than or equal to \$2.00.
\end{itemize}

\subsection{ii. Results for Ingredients, Cost, and Calories}
Using the optimization that minimizes weight of ingredients, the solution has the following results:\\

Objective value: 1.336521\\

\begin{tabular}{|c|c|c|c|c|c|c|c|c|}
	\hline variables & $x_{1}$ & $x_{2}$ & $x_{3}$ & $x_{4}$ & $x_{5}$ & $x_{6}$ & $x_{7}$ & $x_{8}$ \\
	\hline optimal value & 0 & 0 & 53.46083 & 0 & 8.654325 & 71.53693 & 0 & 0 \\
	\hline
\end{tabular} \\

{\it Note:} The units for the objective value is total grams of all ingredients and the variables are in grams per ingredient.\\

Thus we should select 53.46083g of Spinach, 8.654325g of Sunflower Seeds, and 71.53693g of Smoked Tofu. The resulting salad will have the following costs and calories:

\begin{displaymath}
({\$0.50 \over 100\text{g}})x_3 + ({\$0.45 \over 100\text{g}})x_5 + ({\$2.15 \over 100\text{g}})x_6 =
\end{displaymath}
\begin{displaymath}
(\$0.0050 \times 53.46083\text{g}) + (\$0.0045 \times 8.654325\text{g}) + (\$0.0215 \times 71.53693\text{g}) =
\end{displaymath}
\begin{displaymath}
\$1.844292608
\end{displaymath}

\begin{displaymath}
({40\text{ kcal} \over 100\text{g}})x_3 + ({585\text{ kcal} \over 100\text{g}})x_5 + ({120\text{ kcal} \over 100\text{g}})x_6 =
\end{displaymath}
\begin{displaymath}
(0.4\text{ kcal} \times 53.46083\text{g}) + (5.85\text{ kcal} \times 8.654325\text{g}) + (1.20\text{ kcal} \times 71.53693\text{g}) =
\end{displaymath}
\begin{displaymath}
157.85644925\text{ kcal}
\end{displaymath}

Thus, the costs are reduced to \$1.84 and the calories are reduced to 157.86 kcal.

\subsection{iii. Derived Solution}
The minimizing weight of ingredients solution, which we selected for evaluation, was derived by determining that the objective was to minimize both cost and calories in equal parts, but that both of these factors are determined by the weight of ingredients. Since the weight of ingredients affects both cost and calories in the same manner, increased weight equates to increased costs and increased calories, we chose to minimize the weight.

We also ran the minimizing cost and minimizing calories solutions in order to compare the results against our selected solution. The comparison showed us that optimizing for calories resulted in the absolute minimum number of calories possible, whereas the optimization for cost resulted in the absolute minimum for calories. Since the given problem says that we should optimize for both, and given the value comparisons for all three solutions, we determined that optimizing for weight of ingredients resulted in the most equitable solution.

\chapter{Shortest Path Problems}

\section{Part A}\label{part3a}
In this question, our goal is to minimize the path from vertex $a$ to every other vertex. So the code in Lindo is written as follows:
\begin{lstlisting}[language=c]
max a+b+c+d+e+f+g+h+i+j+k+l+m
st
        a      = 0 
        b - a <= 2
        c - a <= 3
        d - a <= 8
        h - a <= 9
        a - b <= 4
        c - b <= 5
        e - b <= 7
        f - b <= 4
        d - c <= 10
        b - c <= 5
        g - c <= 9
        i - c <= 11
        f - c <= 4
        a - d <= 8
        g - d <= 2
        j - d <= 5
        f - d <= 1
        h - e <= 5
        c - e <= 4
        i - e <= 10
        i - f <= 2
        g - f <= 2
        d - g <= 2
        j - g <= 8
        k - g <= 12
        i - h <= 5
        k - h <= 10
        a - i <= 20
        k - i <= 6
        j - i <= 2
        m - i <= 12
        i - j <= 2
        k - j <= 4
        l - j <= 5
        h - k <= 10
        m - k <= 10
        m - l <= 2
\end{lstlisting}

The reason that we maximize $a+b+c+d+e+f+g+h+i+j+k+l+m$ is that we want to get the shortest path for every vertex in one run so that we can get all we need in the final result.\\

From this code we can get the optimal solution as shown in the following table.\\
\begin{tabular}{|c|c|c|c|c|c|c|c|c|c|c|c|c|c|}
	\hline vertex & $a$   &  $b$ & $c$ & $d$ & $e$ & $f$ & $g$ & $h$ & $i$  & $j$ & $k$ & $l$ & $m$   \\
	\hline shortest path from $a$ & 0 & 2 & 3 & 8 & 9 & 6 & 8 & 9 & 8 & 10 & 14 & 15 & 17       \\
	\hline
\end{tabular} \\

\section{Part B}\label{part3b}
Since there is no path from vertex $a$ to vertex $z$, there will be an error when we run the program. And the code we run proves this.
\begin{lstlisting}[language=c]
max a+b+c+d+e+f+g+h+i+j+k+l+m+z
st
        a      = 0 
        b - a <= 2
        c - a <= 3
        d - a <= 8
        h - a <= 9
        a - b <= 4
        c - b <= 5
        e - b <= 7
        f - b <= 4
        d - c <= 10
        b - c <= 5
        g - c <= 9
        i - c <= 11
        f - c <= 4
        a - d <= 8
        g - d <= 2
        j - d <= 5
        f - d <= 1
        h - e <= 5
        c - e <= 4
        i - e <= 10
        i - f <= 2
        g - f <= 2
        d - g <= 2
        j - g <= 8
        k - g <= 12
        i - h <= 5
        k - h <= 10
        a - i <= 20
        k - i <= 6
        j - i <= 2
        m - i <= 12
        i - j <= 2
        k - j <= 4
        l - j <= 5
        h - k <= 10
        m - k <= 10
        m - l <= 2
\end{lstlisting}
 After running this, we get an ``{\it unbounded solution}'' result. This result makes logical sense since no bounding constraints were defined for $z$ and thus it is by definition a completely unbounded variable. Thus, there is no shortest path from vertex $a$ from $z$.
 
\section{Part C}\label{part3c}
To calculate the shortest path from any vertex $n$ to vertex $m$, we can simply reverse the bounds in Part~\ref{part3a}. Which means that we reverse every path from one vertex to another, and calculate the shortest path from vertex $m$. For instance, if the path from $a$ to $b$ is 2, then we reverse this to become the path from $b$ to $a$, which is still 2. In this case we can calculate the shortest path from every vertex $n$ to vertex $m$ in one run.

We can write the code as follows:
\begin{lstlisting}[language=c]
max a+b+c+d+e+f+g+h+i+j+k+l
st
        m      = 0 
        a - b <= 2
        a - c <= 3
        a - d <= 8
        a - h <= 9
        b - a <= 4
        b - c <= 5
        b - e <= 7
        b - f <= 4
        c - d <= 10
        c - b <= 5
        c - g <= 9
        c - i <= 11
        c - f <= 4
        d - a <= 8
        d - g <= 2
        d - j <= 5
        d - f <= 1
        e - h <= 5
        e - c <= 4
        e - i <= 10
        f - i <= 2
        f - g <= 2
        g - d <= 2
        g - j <= 8
        g - k <= 12
        h - i <= 5
        h - k <= 10
        i - a <= 20
        i - k <= 6
        i - j <= 2
        i - m <= 12
        j - i <= 2
        j - k <= 4
        j - l <= 5
        k - h <= 10
        k - m <= 10
        l - m <= 2
\end{lstlisting}

From this code we can get the optimal solution as shown in the following table.\\
\begin{tabular}{|c|c|c|c|c|c|c|c|c|c|c|c|c|c|}
	\hline vertex & $a$   &  $b$ & $c$ & $d$ & $e$ & $f$ & $g$ & $h$ & $i$  & $j$ & $k$ & $l$ & $m$   \\
	\hline shortest path to $m$ & 17 & 15 & 15 & 12 & 19 & 11 & 14 & 14 & 9 & 7 & 10 & 2 & 0 \\
	\hline
\end{tabular} \\

We can see from this table that the shortest path from $a$ to $m$ is 17, which matches the result that we got in Part~\ref{part3a}. Therefore, this method suffices for all requirements and constraints.

\section{Part D}

Because every path has to pass through vertex $i$, we can cut this question into 2 parts. The first part is to calculate the shortest path from every vertex $n$ to vertex $i$, using the similar method we used in Part~\ref{part3c}. The second part is to calculate the shortest path from vertex $i$ to every other vertex. If there's any error when running the codes, then it means we cannot get there by passing through vertex $i$. 

The code for the first part is written as follows:
\begin{lstlisting}[language=c]
max a + b + c + d + e + f + g + h + i + j + k
st
        i      = 0 
        a - b <= 2
        a - c <= 3
        a - d <= 8
        a - h <= 9
        b - a <= 4
        b - c <= 5
        b - e <= 7
        b - f <= 4
        c - d <= 10
        c - b <= 5
        c - g <= 9
        c - i <= 11
        c - f <= 4
        d - a <= 8
        d - g <= 2
        d - j <= 5
        d - f <= 1
        e - h <= 5
        e - c <= 4
        e - i <= 10
        f - i <= 2
        f - g <= 2
        g - d <= 2
        g - j <= 8
        g - k <= 12
        h - i <= 5
        h - k <= 10
        i - a <= 20
        i - k <= 6
        i - j <= 2
        i - m <= 12
        j - i <= 2
        j - k <= 4
        j - l <= 5
        k - h <= 10
        k - m <= 10
        l - m <= 2
\end{lstlisting}

We can see that there is no vertex $l$ and $m$ in this maximizing code for the reason that if we add them in this code there would be an error, which means there is no path from $l$ or $m$ to vertex $i$.

From this code we can get the optimal solution as shown in the following table.\\
\begin{tabular}{|c|c|c|c|c|c|c|c|c|c|c|c|c|c|}
	\hline vertex & $a$   &  $b$ & $c$ & $d$ & $e$ & $f$ & $g$ & $h$ & $i$  & $j$ & $k$ & $l$ & $m$   \\
	\hline shortest path to $i$ & 8 & 6 & 6 & 3 & 10 & 2 & 5 & 5 & 0 & 2 & 15 & N/A & N/A \\
	\hline
\end{tabular} \\

The code for the second part is written as follows:
\begin{lstlisting}[language=c]
max a + b + c + d +e + f + g + h + i + j + k + l + m
st
        i      = 0 
        b - a <= 2
        c - a <= 3
        d - a <= 8
        h - a <= 9
        a - b <= 4
        c - b <= 5
        e - b <= 7
        f - b <= 4
        d - c <= 10
        b - c <= 5
        g - c <= 9
        i - c <= 11
        f - c <= 4
        a - d <= 8
        g - d <= 2
        j - d <= 5
        f - d <= 1
        h - e <= 5
        c - e <= 4
        i - e <= 10
        i - f <= 2
        g - f <= 2
        d - g <= 2
        j - g <= 8
        k - g <= 12
        i - h <= 5
        k - h <= 10
        a - i <= 20
        k - i <= 6
        j - i <= 2
        m - i <= 12
        i - j <= 2
        k - j <= 4
        l - j <= 5
        h - k <= 10
        m - k <= 10
        m - l <= 2
\end{lstlisting}

In contrast to the first part, we have vertex $l$ and $m$ here because there is a shortest path from $i$ to them so that we don't have to erase them.

From this code we can get the optimal solution as shown in the following table.\\
\begin{tabular}{|c|c|c|c|c|c|c|c|c|c|c|c|c|c|}
	\hline vertex & $a$   &  $b$ & $c$ & $d$ & $e$ & $f$ & $g$ & $h$ & $i$  & $j$ & $k$ & $l$ & $m$   \\
	\hline shortest path from $i$ & 20 & 22 & 23 & 28 & 29 & 26 & 28 & 16 & 0 & 2 & 6 & 7 & 9 \\
	\hline
\end{tabular} \\

By adding these two matrices together, we can get the final result as follows:\\
\begin{tabularx}{\textwidth}{|l|X|X|X|X|X|X|X|X|X|X|X|X|X|}

\hline  	& $a$ &  $b$ & $c$ & $d$ & $e$ & $f$ & $g$ & $h$ & $i$  & $j$ & $k$ & $l$ & $m$\\
\hline $a$ &	28 &	30 &	31 &	36 &	37 &	34 &	36 &	24 &	8 &	10 &	14 &	15 &	17\\
\hline $b$ &	26 &	28 &	29 &	34 &	35 &	32 &	34 &	22 &	6 &	8 &	12 &	13 &	15\\
\hline $c$ &	26 &	28 &	29 &	34 &	35 &	32 &	34 &	22 &	6 &	8 &	12 &	13 &	15\\
\hline $d$ &	23 &	25 &	26 &	31 &	32 &	29 &	31 &	19 &	3 &	5 &	9 &	10 &	12\\
\hline $e$ &	30 &	32 &	33 &	38 &	39 &	36 &	38 &	26 &	10 &	12 &	16 &	17 &	19\\
\hline $f$ &	22 &	24 &	25 &	30 &	31 &	28 &	30 &	18 &	2 &	4 &	8 &	9 &	11\\
\hline $g$ &	25 &	27 &	28 &	33 &	34 &	31 &	33 &	21 &	5 &	7 &	11 &	12 &	14\\
\hline $h$ &	25 &	27 &	28 &	33 &	34 &	31 &	33 &	21 &	5 &	7 &	11 &	12 &	14\\
\hline $i$ &	20 &	22 &	23 &	28 &	29 &	26 &	28 &	16 &	0 &	2 &	6 &	7 &	9\\
\hline $j$ &	22 &	24 &	25 &	30 &	31 &	28 &	30 &	18 &	2 &	4 &	8 &	9 &	11\\
\hline $k$ &	35 &	37 &	38 &	43 &	44 &	41 &	43 &	31 &	15 &	17 &	21 &	22 &	24\\
\hline $l$ &	$N/A$ &	$N/A$ &	$N/A$ &	$N/A$ &	$N/A$ &	$N/A$ &	$N/A$ &	$N/A$ &	$N/A$ &	$N/A$ &	$N/A$ &	$N/A$ &	$N/A$\\
\hline $m$ &	$N/A$ &	$N/A$ &	$N/A$ &	$N/A$ &	$N/A$ &	$N/A$ &	$N/A$ &	$N/A$ &	$N/A$ &	$N/A$ &	$N/A$ &	$N/A$ &	$N/A$\\
\hline
\end{tabularx} \\

where $N/A$ stands for there is no path.

\end{document} 
