\documentclass[11pt]{scrreprt}
\usepackage[utf8]{inputenc}
\usepackage{graphicx}
\usepackage{listings} %插入代码
\usepackage{xcolor} %代码高亮
%\usepackage[demo]{graphicx}
\usepackage{caption}
\usepackage{tabularx}
\usepackage[labelformat=simple]{subcaption}
\usepackage{amsmath}

\renewcommand\thesubfigure{(\alph{subfigure})} % see subcaption doc

\lstset{numbers=left, %设置行号位置
	numberstyle=\tiny, %设置行号大小
	keywordstyle=\color{blue}, %设置关键字颜色
	commentstyle=\color[cmyk]{1,0,1,0}, %设置注释颜色
	frame=single, %设置边框格式
	escapeinside=``, %逃逸字符(1左面的键),用于显示中文
	breaklines, %自动折行
	extendedchars=false, %解决代码跨页时,章节标题,页眉等汉字不显示的问题
	xleftmargin=2em,xrightmargin=2em, aboveskip=1em, %设置边距
	tabsize=4, %设置tab空格数
	showspaces=false %不显示空格
}

\title{\textbf{Project 4 Report}}
\subtitle{CS325 Analysis of Algorithms, Summer 2015}
\author{\textsf{\textbf{Project Group 1}}\\
		\textsf{Tingzhi Li}\\
		\textsf{Nicholas Nelson}\\
		\textsf{Chunyang Zhang}}
\date{}

\begin{document}
\maketitle

\chapter{Algorithm Idea}

Simulated annealing optimization is the technique we choose to solve Traveling Salesman Problem. It is one of the classic way to find minimum value of a system. The term annealing originally comes from metallurgy. It is a technique that consist of heating and cooling the material in a controlled manner in order to reduce their defects. In computer science field, simulated annealing optimization mimics the metallurgy one. Basically, our algorithm has a starting temperature which is pretty high and slowly cools down to a ending temperature which is pretty low. The cooling process is done by using a variable \emph{temperature}, for each iteration the \emph{temerature} will be multiplied with a cooling factor that is between 0 and 1. The objective function of this algorithm is to find the global shorest distance that satisfy the requirement of TSP. The input of objective function is a sequence of cities salesman will visit. We start our algorithm with a random sequence. For each iteration, the algorithm randomly change the sequence and check if the new sequence yields a shorter traveling distance. If it does, keep the new sequence meaning update the variabl \emph{bestSequence}. If it does not, this algorithm will accept the new sequence at certain probability. This probability is computed based on the difference between the new distance and previous distance, and the temperature.
The key concept here is as following:

$d = cost_{new} - cost_{prev}$  

if $d < 0$ or $e^{-d/temp} > random(0,1)$

\quad $cost_{prev} = cost_{new}$

In summary, applying annealing technique to TSP consists of following steps:

\begin{itemize}
	\item create a random visiting sequence for initialization
	\item Randomly swap two cities in the sequence and compute the new distance
	\item If the new distance is shorter than previous one, keep the new sequence
	\item If the new distance is longer than previous one, keep the new sequence with certain probability
	\item Update the temperature at each iteration until the temperature cools down to stopping temperature
\end{itemize}

Our implementation is based on the work of Matthew Perry and Richard J. Wagner at University of Michigan. Some of our understand of annealing technique is based on the article written by Emmanuel Goossaert called "Simulated annealing applied to the traveling salesman problem."		 

\section{Overall Design}

\section{Technical Specifications}

\chapter{Tours \& Validation Timings}
\section{Example Instance 1}

\ \ \ The solution of the length is: $ $.

The time it took to obtain these tours is: $ $ minutes.


\section{Example Instance 2}

\ \ \ The solution of the length is: $ $.

The time it took to obtain these tours is: $ $ minutes.


\section{Example Instance 3}

\ \ \ The solution of the length is: $ $.

The time it took to obtain these tours is: $ $ minutes.

\chapter{Competition Tours}
\section{Best Tour for Test 1}

The solution of the length is: $ $.


\section{Best Tour for Test 2}

The solution of the length is: $ $.


\section{Best Tour for Test 3}

The solution of the length is: $ $.


\section{Best Tour for Test 4}

The solution of the length is: $36102$.


\section{Best Tour for Test 5}

The solution of the length is: $ $.


\section{Best Tour for Test 6}

The solution of the length is: $199730$.


\section{Best Tour for Test 7}

We cannot get the solution for this problem in 5 minutes. The reason is that the number of input is very large, and our coding requires to initialize the problem first, therefore we cannot even set up the problem to run so that we cannot get the result of it.

\end{document} 
